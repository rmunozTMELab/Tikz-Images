% =============================================================================
% File:     embedding_models_pgnniv.tex
% Purpose:  Visual schematic of the different embeddings of PGNNIV
% Author:   Rubén Muñoz-Sierra (739163@unizar.es; rmunnoz@iisaragon.es)
% =============================================================================

\documentclass[border=3pt,tikz]{standalone}

% =============================================================================
% PREAMBLE
% =============================================================================

% -----------------------------
% PACKAGES
% -----------------------------
\usepackage{amsmath}
\usepackage{listofitems}
\usepackage[outline]{contour}
\usepackage{xcolor}
\usepackage{tikz}
\usepackage{tikz-3dplot}

% -----------------------------
% TIKZ LIBRARIES
% -----------------------------
\usetikzlibrary{
  arrows.meta,
  3d,
  decorations.pathreplacing
}

% -----------------------------
% COLOR DEFINITIONS
% -----------------------------
\colorlet{myred}{red!80!black}
\colorlet{myblue}{blue!80!black}
\colorlet{mygreen}{green!60!black}
\colorlet{myorange}{orange!70!red!60!black}
\colorlet{mydarkred}{red!30!black}
\colorlet{mydarkblue}{blue!40!black}
\colorlet{mydarkgreen}{green!30!black}

% -----------------------------
% GLOBAL TIKZ STYLES
% -----------------------------
\contourlength{1.4pt}
\tikzset{
    >=latex,
    % General node style
    node/.style={thick, circle, draw=myblue, minimum size=15, inner sep=0.5, outer sep=0.6},
    % Specific node styles for different layers
    node in/.style={node, draw=mygreen!30!black, fill=mygreen!25, minimum size=12},
    node hidden/.style={node, draw=myblue!30!black, fill=myblue!20, minimum size=12},
    node convol/.style={node, draw=myorange!30!black, fill=myorange!20, minimum size=12},
    node out/.style={node, draw=myred!30!black, fill=myred!20, minimum size=12},
    % Connection styles
    connect/.style={thick, mydarkblue},
    connect arrow/.style={-{Latex[length=4,width=3.5]}, thick, mydarkblue, shorten <=0.5, shorten >=1},
    % Node styles for the embedded neural network diagram
    node 1/.style={node in},
    node 2/.style={node hidden},
    node 3/.style={node out}
}
% Helper macro to map layer number to a style (1, 2, or 3)
\def\nstyle{int(\lay<\Nnodlen?min(2,\lay):3)}

% =============================================================================
% DOCUMENT BODY
% =============================================================================
\begin{document}
\begin{tikzpicture}

% =============================
% GLOBAL CONSTANTS & DIMENSIONS
% =============================
\def\objectspace{0.3}
\def\margins{0.4}
\def\globalsquarewidth{21}
\def\globalsquareheigth{11.5}

% =============================================================================
% AUTOENCODER & EXPLANATORY MODEL SCHEMATIC (TOP PART)
% =============================================================================

% -----------------------------
% INPUT BLOCK
% -----------------------------
\def\inputheight{4}
\def\inputwidth{0.4}
\coordinate (input_origin) at ({\margins + 0.2}, {\globalsquareheigth - \margins});
% Draw the block container
\draw [myblue!60, fill=myblue, fill opacity=0.02, rounded corners=3, line width=0.7mm] 
    (input_origin) --++ (0, -\inputheight) --++ (\inputwidth, 0) --++ (0,\inputheight) -- cycle;
% Draw inner horizontal lines
\foreach \i in {1,...,9} {
    \draw[myblue!30, line width=0.3mm] ($(input_origin) + (0.05, -0.4*\i)$) -- ($(input_origin) + (\inputwidth - 0.05, -0.4*\i)$);
}

% -----------------------------
% ENCODER BLOCK
% -----------------------------
\def\encodingwidth{3}
\def\latentspaceheight{1.5}
\pgfmathsetmacro{\encodingheight}{(\inputheight - \latentspaceheight) / 2}
\coordinate (encoding_origin) at ($(input_origin) + (\inputwidth + \objectspace, 0)$);
% Draw the encoder shape
\draw[myorange!60, fill=myorange, fill opacity=0.1, rounded corners=3, line width=0.6mm]
    (encoding_origin) --++ (\encodingwidth, -\encodingheight) --++ (0, -\latentspaceheight) --++ (-\encodingwidth, -\encodingheight) -- cycle;

% -----------------------------
% LATENT SPACE BLOCK
% -----------------------------
\def\latentspacewidth{0.8}
\coordinate (ls_origin) at ($(encoding_origin) + (\encodingwidth + \objectspace, -\encodingheight)$);
% Draw the latent space rectangle
\draw[mydarkgreen!60, fill=mydarkgreen, fill opacity=0.1, rounded corners=3, line width=0.6mm]
    (ls_origin) rectangle ++(\latentspacewidth, -\latentspaceheight);

% -----------------------------
% DECODER BLOCK (PREDICTIVE)
% -----------------------------
\def\decodingwidth{\encodingwidth}
\def\decodingheight{\encodingheight}
\coordinate (decoding_origin) at ($(ls_origin) + (\latentspacewidth + \objectspace, 0)$);
% Draw the decoder shape (dashed)
\draw[myorange!60, fill=myorange, fill opacity=0.1, rounded corners=3, line width=0.6mm, dash pattern=on 10pt off 3pt,]
    (decoding_origin) --++ (\decodingwidth, \decodingheight) --++ (0, -\inputheight) --++ (-\decodingwidth, \decodingheight) -- cycle;

% -----------------------------
% PREDICTIVE OUTPUT BLOCK
% -----------------------------
\def\predictiveoutputwidth{\inputheight}
\def\predictiveoutputheight{\inputheight}
\coordinate (predictive_output_origin) at ($(decoding_origin) + (\decodingwidth + \objectspace, \decodingheight)$);
% Draw the grid inside the block
\foreach \i in {1,...,9} {\draw[myblue!30, line width=0.3mm] (9, 11.1 - \i*0.4) -- (13, 11.1 - \i*0.4);}
\foreach \j in {1,...,9} {\draw[myblue!30, line width=0.3mm] (9 + \j*0.4, 11.1) -- (9 + \j*0.4, 7.1);}
% Draw the block container
\draw [myblue!60,fill=myblue,fill opacity=0.02,rounded corners=2, line width=0.6mm]
    (predictive_output_origin) --++ (\predictiveoutputwidth, 0) --++ (0, -\predictiveoutputheight) --++ (-\predictiveoutputwidth, 0) -- cycle;

% -----------------------------
% EXPLANATORY DECODER BLOCK
% -----------------------------
\def\explanatorywidth{\decodingwidth/2.2}
\def\explanatoryheight{\decodingheight*1.5}
\coordinate (explanatory_origin) at ($(predictive_output_origin) + (\predictiveoutputwidth + \objectspace, -\decodingheight - \latentspaceheight/2)$);
% Draw the explanatory decoder shape
\draw[myorange!60, fill=myorange, fill opacity=0.1, rounded corners=3, line width=0.6mm]
    ($(explanatory_origin) + (0, 0.2)$) --++ (\explanatorywidth, \explanatoryheight) --++ (\explanatorywidth, -\explanatoryheight) --++ (-\explanatorywidth, -\explanatoryheight) -- cycle;

% -----------------------------
% EXPLANATORY OUTPUT BLOCK
% -----------------------------
\def\predictiveoutputwidth{\inputheight}
\def\predictiveoutputheight{\inputheight}
\coordinate (predictive_output_origin) at ($(explanatory_origin) + (\explanatorywidth*2 + \objectspace, \predictiveoutputheight/2 )$);
% Draw the grid inside the block
\foreach \i in {1,...,9} {\draw[myblue!30, line width=0.3mm] ($(predictive_output_origin) + (0, -\i*0.4)$) -- ($(predictive_output_origin) + (4, -\i*0.4)$);}
\foreach \j in {1,...,9} {\draw[myblue!30, line width=0.3mm] ($(predictive_output_origin) + (\j*0.4, 0)$) -- ($(predictive_output_origin) + (\j*0.4, -4)$);}
% Draw the block container
\draw [myblue!60,fill=myblue,fill opacity=0.02,rounded corners=2, line width=0.6mm]
    (predictive_output_origin) --++ (\predictiveoutputwidth, 0) --++ (0, -\predictiveoutputheight) --++ (-\predictiveoutputwidth, 0) -- cycle;

% -----------------------------
% ANNOTATIONS FOR OUTPUTS (STENCILS)
% -----------------------------
% Horizontal connecting lines
\draw[black!50, very thick] (11, 9.3) -- (13.34, 9.3);
\draw[black!50, very thick] (16.02, 9.3) -- (17.91, 9.3);

% Stencil symbol for the predictive output (green)
\draw[->, thick] (10.8, 9.5) --++ (0, 0.4);
\draw[->, thick] (11, 9.3) --++ (0.4, 0);
\draw[->, thick] (10.8, 9.1) --++ (0, -0.4);
\draw[->, thick] (10.6, 9.3) --++ (-0.4, 0);
\draw[mygreen!60,fill=mygreen,fill opacity=0.2,rounded corners=1pt, line width=0.4mm] 
(11, 9.1) --++ (0, 0.4) --++ (-0.4, 0) --++ (0, -0.4) -- cycle;

% Stencil symbol for the explanatory output (red)
\draw[->, thick] (18.13, 9.5) --++ (0, 0.4);
\draw[->, thick] (18.33, 9.3) --++ (0.4, 0);
\draw[->, thick] (18.13, 9.1) --++ (0, -0.4);
\draw[->, thick] (17.93, 9.3) --++ (-0.4, 0);
\draw[myred!60,fill=myred,fill opacity=0.2,rounded corners=1pt, line width=0.4mm] 
(18.33, 9.1) --++ (0, 0.4) --++ (-0.4, 0) --++ (0, -0.4) -- cycle;

% =============================================================================
% PHYSICS-INFORMED DECODING VARIANTS (BOTTOM PART)
% =============================================================================

% -----------------------------
% OUTER DASHED FRAME
% -----------------------------
\pgfmathsetmacro{\outerwidth}{\globalsquarewidth - \margins}
\pgfmathsetmacro{\outerheight}{\globalsquareheigth - \inputheight - \objectspace - \margins - 0.5}
\pgfmathsetmacro{\lowermarginrectangle}{\margins + 1.}
% Draw the main rectangular part of the frame
\draw[rounded corners=15pt, line width=0.7mm, dash pattern=on 10pt off 3pt, opacity=0.3]
    (10.1, \outerheight)
    -- (\outerwidth, \outerheight)
    -- (\outerwidth, \lowermarginrectangle+0.8)       
    -- (\margins, \lowermarginrectangle+0.8)           
    -- (\margins, \outerheight)  
    -- (3.9, \outerheight);
% Draw the curved top part of the frame
\draw[line width=0.7mm, dash pattern=on 10pt off 3pt, opacity=0.3]
    (4, \outerheight) arc[start angle=270,end angle=303,radius=5.5cm]
    (10, \outerheight) arc[start angle=270,end angle=237,radius=5.5cm];

% -----------------------------
% INNER RECTANGULAR CONTAINERS
% -----------------------------
\def\numrects{3}
\def\squaresep{0.5}
% Calculate dimensions based on global constants
\pgfmathsetmacro{\totalgap}{(\numrects - 1) * \squaresep}
\pgfmathsetmacro{\rectwidth}{(\outerwidth - \totalgap - 3*\margins)/\numrects}
\pgfmathsetmacro{\ystart}{2*\lowermarginrectangle - 0.1}
\pgfmathsetmacro{\yend}{\outerheight - \margins}
% Loop to draw the three containers
\foreach \i in {0,1,2} {
    \pgfmathsetmacro{\x}{2*\margins + \i * (\rectwidth + \squaresep)}
    \draw[fill=blue!20, draw=blue!80, rounded corners=15, line width=0.7mm, opacity=0.5]
        (\x, \ystart) rectangle ++(\rectwidth, \yend - \ystart);
}

% -----------------------------
% CONTENT FOR DECODING VARIANTS
% -----------------------------
\foreach \i in {0,1,2} {
    % Calculate x-position for the content inside each rectangle
    \pgfmathsetmacro{\x}{
        2*\margins 
        + (\rectwidth - \encodingwidth)/2 
        + \i * (\rectwidth + \squaresep)
    }

    % --- NOTE: Optional (a), (b), (c) labels. Currently commented out. ---
    % \node at (\x + \encodingwidth/2, \ystart - 0.7) 
    % {\fontsize{12}{14}\selectfont
    %     \ifnum\i=0 (a) \else \ifnum\i=1 (b) \else \ifnum\i=2 (c) \else (d) \fi \fi \fi
    % };

    % Place content (equations or diagrams) in the center of each container
    \node at (\x + \encodingwidth/2, \outerheight - \encodingheight - \latentspaceheight/2) 
    {\fontsize{12}{14}\selectfont
        % Case 1: Fourier Inverse Transform
        \ifnum\i=0 
            $
                \begin{array}{c}
                \mathcal{F}^{-1} \\[0.5em]
                \boldsymbol{u}(\boldsymbol{r}; \mathbf{x}) = \sum_{k=0}^n u_k(\mathbf{x}) e^{2\pi i \boldsymbol{\xi}_n \cdot \boldsymbol{r}}
                \end{array}
            $
        % Case 2: Proper Orthogonal Decomposition (POD)
        \else\ifnum\i=1 
            $
                \begin{array}{c}
                \mathcal{POD} \\[0.5em]
                \boldsymbol{A} \approx \boldsymbol{A}_k = \boldsymbol{U} \boldsymbol{\Sigma}_k \boldsymbol{V}^T
                \end{array}
            $
        % Case 3: Neural Network Diagram
        \else\ifnum\i=2
            \begin{tikzpicture}[scale=0.7]
                \readlist\Nnod{2,3,4,4} % N_nodes per layer
                \foreachitem \N \in \Nnod { % Loop over layers
                    \def\lay{\Ncnt} % Current layer index
                    \pgfmathsetmacro\prev{int(\Ncnt-1)} % Previous layer index
                    \foreach \i [evaluate={\y=\N/2-\i+0.5; \x=\lay; \n=\nstyle;}] in {1,...,\N} { % Loop over nodes
                        % Draw node
                        \node[node \n, outer sep=0.6] (N\lay-\i) at (\x,\y) {};
                        % Draw connections to previous layer
                        \ifnum\lay>1 
                            \foreach \j in {1,...,\Nnod[\prev]} {
                                \draw[connect, white, line width=1.2] (N\prev-\j) -- (N\lay-\i); % White outline for clarity
                                \draw[connect] (N\prev-\j) -- (N\lay-\i);
                            }
                        \fi
                    }
                }
            \end{tikzpicture}
        % Fallback case (currently unused)
        \else 
            $\mathcal{FFT}$ 
        \fi\fi\fi
    };
}

% =============================================================================
% LABELS FOR MAIN COMPONENTS
% =============================================================================
\node at (0.8, 11.5) {\textbf{Input}};
\node at (2.8, 9.1) {\textbf{Encoding}};
\node at (7.2, 9.1) {\textbf{Decoding}};
\node[align=center] at (5, 10.5) {\textbf{Latent} \\ \textbf{space}};
\node at (14.7, 9.3) {\textbf{Explanatory}};

\end{tikzpicture}
\end{document}